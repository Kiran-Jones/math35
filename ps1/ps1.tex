\documentclass[12pt]{article}
\usepackage{fullpage,enumitem,amsmath,amssymb,graphicx,amsthm,fancyhdr}
\usepackage{../../../macros}

\SetupAssignment{Math 35}{25X}{Homework 1}
\author{Kiran Jones \\
  \texttt{kiran.p.jones.27@dartmouth.edu}}
\date{Due: July 2, 2025}

\pagestyle{fancy}
\lhead{Math 35 – HW 1}\rhead{Kiran Jones}\cfoot{\thepage}
\setlength{\headheight}{14.49998pt}
\setlength{\headsep}{10pt}


\begin{document}
  \maketitle
  \thispagestyle{empty}
  \noindent
  \rule{\linewidth}{0.4pt}
  \newpage
  
  
  \section*{Problem 6}
  \setcounter{page}{1}
  Prove that the reciprocal of an irrational number is an irrational number.

  \begin{proof}
    Let $x$ be some irrational number. Assume that its reciprocal, $\frac{1}{x} = \frac{p}{q}$ is a rational number for some integers $p, q$ with $p, q \neq 0$. If $p$ or $q = 0$, then $x$ cannot be irrational: either $x = 0$ or $x$ is undefined. Solving for $x$ gives us:
    \[
      x = \frac{q}{p}.
    \]
    This contradicts the assumption that $x$ is irrational. Therefore, the reciprocal of an irrational number must also be irrational.
  \end{proof}
  \newpage


  \section*{Problem 12}
  Prove that $\sqrt{2} + \sqrt{3}$ is irrational.

  \begin{proof}
    Let $x = \sqrt{2} + \sqrt{3}$. Assume that $x$ is rational, i.e. $x \in \mathbb{Q}$. Then we can rearrange the equation to isolate one of the square roots: $\sqrt{3} = x - \sqrt{2}$.
    Squaring both sides gives us:
    \[
      3 = (x - \sqrt{2})^2 = x^2 - 2x\sqrt{2} + 2.
    \]
    Rearranging this gives us:
    \[
      2x\sqrt{2} = x^2 - 1.
    \]
    Dividing both sides by $2x$ (as $x \neq 0$) gives us:
    \[
      \sqrt{2} = \frac{x^2 - 1}{2x}.
    \]
    Since $x$ is rational, both $x^2 - 1$ and $2x$ are rational numbers, and thus their quotient $\frac{x^2 - 1}{2x}$ is also rational. This implies that $\sqrt{2}$ is rational, which contradicts the fact that $\sqrt{2}$ is irrational. Therefore, our assumption that $x = \sqrt{2} + \sqrt{3}$ is rational must be false, and $\sqrt{2} + \sqrt{3}$ is irrational.
  \end{proof}
  \newpage


  \section*{Problem 20}
  Use the properties of the ordered field to prove the following: if $x < y$ and $z > 0$, then $xz < yz$.

  \medskip
  \noindent\textbf{Ordered Field Properties}
  \begin{enumerate} 
    \item If $x > 0$ and $y > 0$, then $x + y > 0$.
    \item If $x > 0$ and $y > 0$, then $xy > 0$.
    \item $x < y$ if and only if $y - x > 0$.
  \end{enumerate}

  \begin{proof}
    Let $x, y, z \in F$, where $F$ is an ordered field. Using OF-P3, as $x < y$, then $y - x > 0$. Let $a$ equal $y - x$. Then, we have $a > 0$. Using OF-P2, as $a > 0$ and $z > 0$, we can combine these two inequalities to get $az > 0$. Expanding $az$ gives us $az = (y - x)z = yz - xz$, or $yz - xz > 0$. Applying OF-P3, we can rearrange this inequality to be $xz < yz$. Thus, if $x < y$ and $z > 0 $, then $xz < yz$.
  \end{proof}


\end{document}